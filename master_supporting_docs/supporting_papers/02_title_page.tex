\title{\vspace{-2cm}US-China decoupling and the euro area: assessment from a Global Production Networks Macroeconomic Model}
\author{\hspace{-1.2cm}
  {\fontsize{12.5}{13}\selectfont
    \begin{minipage}[t]{0.25\textwidth}
      \centering
      Pablo Aguilar \\ \emph{European Central Bank}
    \end{minipage}
    \begin{minipage}[t]{0.25\textwidth}
      \centering
      Rosi D. Chankova \\ \emph{University of Oxford}
    \end{minipage} 
     \begin{minipage}[t]{0.25\textwidth}
      \centering
      Matthieu Darracq \\ \emph{European Central Bank}
    \end{minipage}
     \begin{minipage}[t]{0.25\textwidth}
      \centering
      Alistair Dieppe \\ \emph{European Central Bank}
    \end{minipage}
    \hfill \break
    \hspace{0.01\textwidth} % Adjust horizontal spacing if needed
    \\[2em]
    \fontsize{12.5}{13}\selectfont
    \begin{minipage}[t]{0.3\textwidth}
      \centering
      Rubén Domínguez-Díaz \\ \emph{Banco de España}
    \end{minipage}
    \begin{minipage}[t]{0.24\textwidth}
      \centering
      José-Elías Gallegos \\ \emph{Banco de España}
    \end{minipage}
    \hspace{0.01\textwidth} % Adjust horizontal spacing if needed
    \begin{minipage}[t]{0.20\textwidth}
      \centering
      Javier Quintana \\ \emph{Banco de España}
    \end{minipage}
  }
  \thanks{Contact person \href{mailto:pablo.aguilar\_garcia@ecb.europa.eu}{pablo.aguilar\_garcia@ecb.europa.eu}. We thank Antonio Eugenelo for excellent research assistance. We thank internal colleagues for helpful comments and discussions. 
  The views expressed are those of the authors and do not necessarily represent the views of the Banco de España, the European Central Bank, and the Eurosystem.}
}

\date{\vspace{0.5cm}\today}                       
\begin{titlepage}
\maketitle
\vspace{-1cm}  % ← tweak this value as needed
\begin{center}
\Large\textbf{Preliminary work}
\end{center}

\noindent	
\begin{abstract}
We study the short-run macroeconomic effects of trade protectionism in a globally integrated economy with sectoral production linkages. Building on a multi-country, multi-sector New Keynesian model, we incorporate country- and sector-specific tariffs to evaluate their impact on output, inflation, and cross-border spillovers. In a stylized US–China trade conflict involving a 50\% bilateral tariff on manufacturing inputs, we find that GDP falls by 2.0\% in the US and 2.5\% in China, while US inflation rises by over 3.5 percentage points. Input–output linkages amplify these effects, accounting for 0.5 percentage points of the contraction and inflation. Under higher trade elasticities, losses deepen for the US and China but generate modest gains for the euro area through trade diversion. The results highlight the importance of production networks and substitution patterns in shaping the global transmission of trade policy shocks.
\end{abstract}

\textbf{Keywords:} Tariffs, Trade, DSGE, Multicoutry, Networks. 

\noindent \textbf{JEL Classifications:}  E31, E32, E52, E70.

\end{titlepage}