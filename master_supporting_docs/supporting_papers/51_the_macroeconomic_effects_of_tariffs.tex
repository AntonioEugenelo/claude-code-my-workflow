\section{The Macroeconomic Effects of Liberation day, PRELIMINARY}\label{sec:5}
\subsection{US tariffs}\label{sec:US_tariffs}
%We assume that 
To quantify the macroeconomic consequences of tariff policy, we simulate three distinct scenarios in our multi-country New Keynesian open-economy model with input–output linkages. Each scenario explores how varying degrees of trade protectionism affect key macroeconomic aggregates across the United States, the Euro Area (EA), China, and the rest of the world (ROW). Scenario 1 assumes that the US imposes a 54\% tariff on Chinese imports, while keeping tariffs at 0\% on EA and ROW, and importantly, assuming no retaliatory measures from trade partners. Scenario 2 raises the stakes, with the US applying a 20\% tariff on EA, a 145\% tariff on China, and a 25\% tariff on ROW, again without retaliation. Scenario 3 mirrors Scenario 2 in tariff magnitudes but incorporates full retaliation: EA, China, and ROW impose tariffs of equal size against US exports. Together, these exercises reveal how the interplay between direct trade barriers, supply-chain disruptions, and exchange-rate dynamics shapes the transmission of tariffs through the global economy.\par

In Scenario 1, the targeted US tariff on China triggers an asymmetric response across countries (Figure \ref{fig:tariff1}). Real GDP declines in both the US and China, both around 0.4 percentage points on impact, reflecting China’s heavy exposure to US demand. The US contraction, though initially more severe, gradually dissipates over eight quarters as domestic substitution and improved net trade partially offset the initial shock, while Chinese growth faces a slower recovery. Inflation rises sharply in the US (peaking at around 0.7 percentage points increase) due to cost-push pressures, while China experiences a more moderate inflationary response. The EA and ROW remain largely insulated in this scenario, displaying negligible output or price changes. US real exports contract modestly, while Chinese exports fall more steeply, reflecting the direct impact of tariffs. The US net trade balance improves markedly, supported by reduced imports, whereas China’s deteriorates.\par

\begin{figure}[H]
  \centering
    \includegraphics[width=\linewidth]{figures/Tariff_1_Overview.png}
    \caption{IRF to a tariff shock (scenario 1: 54\% on China, 0\% on EA and ROW - no retaliation).}
    \label{fig:tariff1}
\end{figure}

Scenario 2 introduces broad-based tariffs without retaliation, considerably amplifying the macroeconomic fallout (Figure \ref{fig:tariff2}). Real GDP in the US contracts sharply (by nearly –1.6 percentage points on impact) as higher import costs and disrupted input supply chains weigh on activity. Inflation rises around 2.5 percentage points, reflecting the pass-through of production cost increases into consumer prices. China also experiences noticeable output declines, while the EA and the ROW sees modest reductions in output and increases in inflation. The US net trade balance improves around 2 percentage points, as imports collapse and exports decline more moderately. \par

\begin{figure}[H]
    \centering
    \includegraphics[width=\linewidth]{figures/Tariff_2_Overview.png}
    \caption{IRF to a tariff shock (scenario 2: 20\% on EA, 145\% on China and 25\% on ROW - no retaliation).}
    \label{fig:tariff2}
\end{figure}

\subsection{Retaliation}
Scenario 3 incorporates full retaliation by EA, China, and ROW against US exports (Figure \ref{fig:tariff3}). The resulting macroeconomic costs for the US therefore are significantly larger and more persistent. US real GDP falls by 2.5 percentage points on impact, with a slow and partial recovery over the simulation horizon. China, EA, and ROW also experience output losses (–1.5 to –0.2 percentage points), driven by both demand destruction and the fragmentation of global supply chains. Inflation rises strongest in the US, while China, the EA and the ROW face significantly smaller increases. Export volumes from the US collapse (around –2.2 percentage points), and the net trade balance improves less than in the non-retaliation case.\par

\begin{figure}[H]
  \centering
    \includegraphics[width=\linewidth]{figures/Tariff_3_Overview.png}
    \caption{IRF to a tariff shock (scenario 3: 20\% on EA, 145\% on China and 25\% on ROW - full retaliation).}
    \label{fig:tariff3}
\end{figure}

Several themes emerge across these scenarios. First, tariffs are unambiguously contractionary for the US and its main trading partners, especially when retaliation is present. Second, the inflation response is highly sensitive to the interaction between supply-side cost increases and demand-side contractions: in the absence of retaliation, tariffs drive sharp inflation; with retaliation, inflationary pressures are dampened by collapsing external demand. Third, exchange rate dynamics play a pivotal role in external adjustment. In non-retaliatory settings, US tariffs trigger dollar depreciation, improving competitiveness; under retaliation, the dollar appreciates, reflecting financial market risk aversion and exacerbating the drag on net exports. Finally, global spillovers matter: even when tariffs are targeted, as in Scenario 1, production networks transmit shocks beyond the directly affected regions.\par

Overall, the simulations highlight that the macroeconomic effects of tariffs depend crucially on both the breadth of the protectionist measures and the international policy response. Selective tariffs with no retaliation, as in Scenario 1, generate modest contractions, manageable inflation, and trade rebalancing. Broad-based tariffs without retaliation, as in Scenario 2, unleash large supply-side shocks that amplify inflation and depress output, despite an improved trade balance. When retaliation is introduced, as in Scenario 3, the growth costs multiply, and the trade gains evaporate, leaving countries worse off across most dimensions. Crucially, the presence of dense international production networks amplifies the persistence and magnitude of these effects, underscoring the need to consider not only bilateral trade flows but also the upstream and downstream linkages that condition the macroeconomic transmission of tariffs.\par


% \begin{figure}[ht]
%   \centering
%     \includegraphics[width=\linewidth]{figures/Combined_IRFs_by_Tariffs.png}
%     \caption{IRF to a tariff shock (under US-China trade war).}
%     \label{fig:IRF1}
% \end{figure}



