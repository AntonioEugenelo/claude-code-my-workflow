\subsection{Results}\label{sec:quantitative_analysis_results}
In this section, we first analyze the dynamics of EA variables, with a focus on how these dynamics are shaped by IO linkages. Second, we analyze the contribution of production networks to inflation dynamics through a series of counterfactuals. Third, we explore how heterogeneity in production structures gives rise to differential inflation dynamics across countries. %Finally, we derive implications for monetary policy.   

The energy price shock we analyze is structured as follows. In both the model and the data, the energy mining sector of the ROW extracts the main energy products.\footnote{In our data, this corresponds with the Mining and quarrying of energy products sectors, which accounts for sections B.5 and B.6 in the ISIC, Rev.4 classification.} These energy goods are then sold to EA firms that primarily belong to the energy sectors Coke and petrol refining and Electricity sector.\footnote{In our data, these correspond with sections C.19 and D.35 in the ISIC, Rev.4 , classification respectively.}  After being processed by these sectors, energy goods are then supplied to households as consumption goods, and to the remaining sectors of the economy as energy intermediate goods used in the production process.

In line with the previous reasoning, we consider a $10\%$ increase in the price wedge $\tau_{klj}$ between the price charged by the energy mining sector located RoW and the price paid by EA firms. Formally, we set $k=\text{\{ES,DE,FR,IT,REA\}}$, $l=\text{ROW}$ and $j= \text{energy mining}$. 