\section{Introduction}

The international economy is shaped by complex production networks and cross-border value chains that tightly link regional developments to global outcomes. Recent shocks—from the war in Ukraine and the subsequent energy crisis to the renewed use of trade restrictions—have underscored how regional disruptions can propagate globally through supply chains and trade relationships. In this context, macroeconomic fluctuations are no longer contained within national borders: region- or sector-specific disturbances often trigger indirect but economically significant spillovers. Understanding such dynamics requires analytical frameworks that can jointly account for domestic impacts and global transmission channels driven by production interdependencies and policy heterogeneity.



To address these challenges, we extend the multi-country, multi-sector New Keynesian general equilibrium model developed in \citet{Aguilar2025} to incorporate country- and sector-specific tariffs. The model features detailed input–output linkages and bilateral trade flows across four economies—the euro area, the United States, China, and the rest of the world—and disaggregates production into 44 sectors. Firms combine domestic labor with both locally sourced and imported intermediates, generating national and international production chains. It incorporates nominal rigidities in prices and wages, heterogeneous monetary policy regimes, and a granular external sector. The calibration draws on OECD and Eurostat input–output tables for labor shares and trade flows, and sector-specific price adjustment frequencies from \citet{Gautier2024}, to reflect heterogeneity in price rigidity and external exposure. Our contribution focuses on embedding trade policy shocks—specifically tariffs—into this structure to assess their macroeconomic and international transmission effects.

We apply this framework to examine the macroeconomic effects of a bilateral trade conflict between the United States and China, modeled as a 50 percent reciprocal tariff on manufacturing inputs. The results indicate that tariffs are both inflationary and contractionary in the short run. Real GDP declines by 2.0 percent in the US and 2.5 percent in China, while US inflation rises by over 3.5 percentage points over a three-year horizon. Production networks amplify these effects: approximately 0.5 percentage points of both the output contraction and inflation increase are attributable to cost spillovers across sectors. In the absence of such linkages, the responses would be notably smaller and less persistent. Under higher trade elasticities, trade diversion toward third countries exacerbates the impact on the US and China—by an additional 10 percent—while modestly benefiting the euro area, where GDP increases by 0.3 percentage points.

By embedding tariffs within a fully networked, open-economy New Keynesian model, this paper provides new theoretical and quantitative insights into the global transmission of trade shocks. The findings highlight how sectoral production structures, trade substitution elasticities, and monetary policy interactions jointly shape the inflation-output trade-off in response to protectionist measures—insights that are essential for assessing the international implications of trade policy in a fragmented global economy.


