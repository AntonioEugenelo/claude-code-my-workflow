\paragraph{Related literature}

This paper contributes to the strands of the literature at the intersection of production networks, international macroeconomics, and trade policy, with a specific focus on the short-run macroeconomic effects of tariff shocks in a networked global economy.

A first line of research examines how shocks propagate through production networks. Seminal contributions by \citet{acemoglu2012}, \citet{gabaix2011}, and \citet{Baqaee2020} show how granular or sector-specific shocks can have aggregate effects when input–output linkages are present. These models, however, largely assume flexible prices and abstract from nominal frictions. More recent work, such as \citet{pasten2020} and \citet{Rubbo2023}, embeds production networks into New Keynesian environments with nominal rigidities. Yet, these contributions are limited to closed economies and cannot capture international trade linkages or cross-border spillovers. Our paper extends this literature by embedding tariff shocks into a multi-country, multi-sector model with rich IO structures and nominal frictions.

A second strand of the literature investigates how nominal rigidities and intermediate inputs influence the transmission of monetary policy. \citet{nakamura2010} show that sectoral heterogeneity in price stickiness and the use of intermediate inputs amplify the real effects of monetary shocks. \citet{huang2006} and \citet{HUANG_LIU_2004} highlight that such frictions increase inflation persistence. Our framework replicates these mechanisms but focuses instead on trade shocks—specifically tariffs—showing how international IO linkages and heterogeneous policy regimes jointly shape the inflation-output trade-off in response to protectionism.

We also contribute to the emerging literature that embeds IO linkages in open economy settings. \citet{baqaee24} study shock transmission in a multi-country IO framework, but abstract from nominal rigidities and monetary policy. \citet{comin2023} develop a small open economy model with nominal rigidities and focus on capacity constraints. \citet{ernst2023} examine environmental policy in a multi-country production network under flexible prices, while \citet{andrade2023aggregate} introduce a three-sector New Keynesian open economy model à la \citet{Gali2005} to study productivity shocks. Relative to this work, our model features richer cross-country heterogeneity and explicitly models trade-policy shocks with sectoral granularity.

Finally, our paper is closely related to a growing body of work that studies the macroeconomic effects of tariffs. Recent contributions have examined tariffs in global trade models {to be updated with recent publications}, as well as in New Keynesian settings with aggregate dynamics {to be updated with recent publications}. We build on this literature by quantifying the inflationary and contractionary effects of tariffs in a model that features both production networks and country- and sector-specific frictions, thereby providing a structural account of how trade protectionism propagates globally in the short run.