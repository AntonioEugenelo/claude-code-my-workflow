\documentclass{beamer}
\usetheme{Madrid}
\useoutertheme{miniframes} % For navigation dots
\useinnertheme{circles}
\usecolortheme{default}

\usepackage[utf8]{inputenc}
\usepackage{lmodern} % better font rendering
\usepackage{amsmath, amssymb}
\usepackage{graphicx}
\usepackage{tcolorbox}

\newif\ifinappendix
\inappendixfalse


\setbeamertemplate{itemize item}{\small$\blacktriangleright$}
\setbeamertemplate{itemize subitem}{\footnotesize$\bullet$}

\author[Antonio Eugenelo]{Antonio Eugenelo}
\institute[Oxford]{University of Oxford}


% Hide navigation symbols at the bottom right
\setbeamertemplate{navigation symbols}{}

\setbeamertemplate{itemize item}{\small$\blacktriangleright$}
\setbeamertemplate{itemize subitem}{\footnotesize$\bullet$}

\makeatletter
\setbeamertemplate{footline}{
  \leavevmode%
  \hbox{%
  \begin{beamercolorbox}[wd=.2\paperwidth,ht=2.25ex,dp=1ex,center]{author in head/foot}%
    \usebeamerfont{author in head/foot}\insertshortauthor
  \end{beamercolorbox}%
  \begin{beamercolorbox}[wd=.5\paperwidth,ht=2.25ex,dp=1ex,center]{title in head/foot}%
    \usebeamerfont{title in head/foot}\insertshorttitle
  \end{beamercolorbox}%
  \begin{beamercolorbox}[wd=.3\paperwidth,ht=2.25ex,dp=1ex,right]{date in head/foot}%
    \usebeamerfont{date in head/foot}\insertshortdate{}%
    \ifinappendix%
\hspace*{2em}
  \phantom{\insertframenumber{} / \inserttotalframenumber}
    \else%
    \hspace*{2em}
    \insertframenumber{} / \inserttotalframenumber\hspace*{2ex} 
    \fi%
  \end{beamercolorbox}}%
  \vskip0pt%
}
\makeatother


\title{Government spending in Disaggregated Economies}
\author{Antonio Eugenelo}
\institute{University of Oxford \\ MPhil in Economics}
\date{June 2025}


\begin{document}

% Title slide
\begin{frame}
  \titlepage
\end{frame}

% Current section
\AtBeginSection[ ]
{
\begin{frame}{Outline}
    \tableofcontents[currentsection]
\end{frame}
}


\section{Introduction}
\begin{frame}{Why Study Multisectoral Policy?}
  \begin{itemize}
    \item Recent large asymmetric shocks (2020 pandemic, 2022 energy‐price spike) highlight the role of sectoral heterogeneity in aggregate dynamics.
    \item Heterogeneity arises from:
      \begin{itemize}
        \item Differential exposure across sectors (Guerrieri et al., 2022; Baqaee \& Farhi, 2022)
        \item Sector‐specific responses to shocks (Pasten et al., 2020)
      \end{itemize}
    \item Implications for optimal policy design (Aoki, 2001; Rubbo, 2023; Cox et al., 2024).
    \item Need to understand the joint design of monetary and fiscal policy in a multi‐sector framework.
  \end{itemize}
\end{frame}

% Slide 2: Findings of Cox et al. (2024)
\begin{frame}{Cox \textit{et al.} 2024}
  \begin{itemize}
    \item Model setup:
      \begin{itemize}
        \item Multi‐sector New‐Keynesian economy
        \item Planner chooses sector‐specific government purchases and an aggregate monetary instrument.
      \end{itemize}
    \item Main result:
      \begin{itemize}
        \item \textbf{Fiscal policy} – should be targeted at the sectoral level.
        \item \textbf{Monetary policy} – should prioritize aggregate output and inflation stabilization.
      \end{itemize}
    \item Key assumptions:
      \begin{enumerate}
        \item Public demand does not enter firms’ private‐good pricing problem (independent prices).
        \item Planner freely chooses both aggregate and sectoral government spending.
      \end{enumerate}
  \end{itemize}
\end{frame}

% Slide 3: Contributions of This Thesis
\begin{frame}{Contributions}
  I relax two critical assumptions in Cox \textit{et al.} (2024):
  \begin{enumerate}
    \item \textbf{Partial pass‐through of government demand:}
      \begin{itemize}
        \item Introduce parameter $\kappa\in[0,1]$ capturing the degree of pass‐through of inelastic government purchases into firms’ private‐good pricing problem.
        \item Nest Cox \textit{et al.} (2024) as special case $\kappa=0$.
        \item Reconcile strong fiscal‐price effects (Ben Zeev \& Pappa, 2017; Barattieri et al., 2023) with weak evidence of procyclical markups (Nekarda \& Ramey, 2013).
      \end{itemize}
  \end{enumerate}
\end{frame}

\begin{frame}{Contributions}
\begin{enumerate}
\setcounter{enumi}{1} 
    \item \textbf{Exogenous aggregate provision of government goods:}
      \begin{itemize}
        \item Allow deviations of aggregate real government good $\bar G$ from steady state to be exogenously set.
        \item Retain flexible, optimal allocation of this exogenous aggregate across sectors.
        \item Capture politically driven constraints on overall government spending.
      \end{itemize}
  \end{enumerate}
\end{frame}




\section{Model Overview}
\begin{frame}{Building Blocks}
\begin{itemize}
\item $K$ sectors, Calvo pricing with heterogenous stickiness $\alpha_k$.
\item Representative household: utility over private ($C_t$) and public ($G_t$) Cobb‑Douglas composites obtained from CES aggregation of firm-level goods:
$$U=\sum_t\beta^t\Bigl[(1-\chi)\frac{C_t^{1-\sigma}}{1-\sigma}+\chi\frac{G_t^{1-\sigma}}{1-\sigma}-\sum_k\nu_k\frac{N_{k,t}^{1+\phi}}{1+\phi}\Bigr]$$
\item The government inelastically buys goods with degree of price pass‑through $\kappa$.
\item Monetary authority chooses nominal rate; fiscal authority allocates $G_{k,t}$ under the balanced budget assumption.
\end{itemize}
\end{frame}

\begin{frame}{Two New Ingredients}
\begin{block}{CRRA preferences ($\sigma>1$)}
\begin{itemize}
\item Precautionary savings and consumption‑insurance motives are strengthened.
\item Aggregate fluctuations feed into sectoral labour supply decisions.
\end{itemize}
\end{block}
\begin{block}{Pass‑through parameter $\kappa$}
\begin{itemize}
\item $\kappa=0$: traditional assumption, public demand does not affect marginal cost.
\item $\kappa>0$: government purchases steepen sectoral Phillips curves; firms partially internalise inelastic demand.
\end{itemize}
\end{block}
\end{frame}

\begin{frame}{Changes in key equations}
    \begin{itemize}
        \item The labour supply condition gains an additional term:
    \end{itemize}
    \[\omega_{c,k}(1-\chi)\frac{\color{red}{C_t^{\sigma-1}}}{C_{k,t}}\, \frac{W_{k,t}}{P_{k,t}}\;=\;\nu_k\, N_{k,t}^\varphi.\]
    \begin{itemize}
        \item While the log-linearised Phillips curve becomes:
    \end{itemize}
    \[\pi_{k,t}=\beta \pi_{k,t+1} + \lambda^ {\color{red}{'}}_k\left[ \psi_{k,t} - a_{k,t} + {\color{red}{\delta\, g_{k,t}}} \right]\]
    \begin{itemize}
        \item Where $\omega_{c,k}$ is the k-th sector weight in the consumption Cobb-Douglas aggregator, $\theta$ is the consumer price elasticity, $\delta\;=\;\frac{\kappa}{\theta-1}\tfrac{\bar G}{\bar C}$, and $\lambda_k^{'} = \lambda_k \frac{1}{1-\delta}$. For $\theta=6$, and $\chi_k^*=1$, $\lambda_k^{'}\,=\, 1.25\lambda_k$
    \end{itemize}
\end{frame}

\section{Policy with $\sigma \neq 1$ \& $\kappa \neq 0$}
\begin{frame}{Optimal Fiscal Policy Rule}
Let us briefly see the updated welfare objective (changes in red):
\begin{align*}\small
- \frac{1}{2} \sum_{k} \mu_k \Biggl(&
    \left[(1+\varphi)\,y_{k,t}^{2} 
    + \frac{\theta (1-\chi)}{\lambda_k}\,\pi_{k,t}^{2} + \chi_k^* \, (g_{k,t} - y_{k,t})^2\right] \notag  \\ 
    &    {\color{red}{+ (\sigma-1)\Biggl[
    (1-\chi_k)\,\omega_{c,k} \left(\frac{y_{k,t}}{1 - \chi_k}- \chi_k^*\,g_{k,t}\right)^2 
    + \omega_{g,k}\,\chi_k\,g_{k,t}^{2}
\Biggr]}}\Biggr)
\end{align*}
\vspace{-10pt}
\begin{itemize}
\item Fiscal policy balances three motives:\\
\textbf{(i) Allocation} of public goods\\
\textbf{(ii) Stabilisation} of sectoral gaps\\
\textbf{(iii) Insurance} against aggregate risk ($\sigma-1$ term)
\end{itemize}
\end{frame}

% Slide 1: Solution
\begin{frame}{Optimal Fiscal Policy Rule}
  \begin{itemize}
    \item The optimal fiscal policy takes the form:
    \begingroup
    \small
    \[
      g_{k,t}
      = \frac{H_{k,t}}{X_{k,t}}\,y_{k,t}
      + \frac{J_{k,t}}{X_{k,t}}\,a_{k,t}
      - \frac{\theta}{\lambda_k\,X_{k,t}}\,\pi_{k,t}
      + \frac{\sigma - 1}{(1-\chi)^{1/\sigma}X_{k,t}}
        \sum_k \mu_k\lambda_k\,\phi_{k,t}^\pi
    \]
    \endgroup
    \vspace{0.5ex}
    \item $\phi_{k,t}^\pi$ is a sectoral index negatively correlated with sectoral output and government spending
    \item $H_{k,t}, J_{k,t}, X_{k,t}>0$ are model‐derived constants
  \end{itemize}
\end{frame}

% Slide 2: Results
\begin{frame}{Optimal Fiscal Policy Rule}
  \textbf{1. The overall coefficient on $y_{k,t}$.}
  Including the contribution from $\phi_{k,t}^\pi$, and setting $\kappa=1$, yields:
  \[
    \Bigl[\tfrac{1}{(\theta-1)(1-\chi_k)\varphi+1}\Bigr]
    \;\times\;
    \Bigl[\tfrac{1}{1-\chi_k} + \varphi
      + (\sigma - 1)\tfrac{\omega_{c,k}}{1-\chi_k}
      - \tfrac{(\theta-1)(1-\chi_k)\varphi}{(1-\chi_k)\lambda_k}\Bigr]
  \]
  This coefficient is negative if and only if:
  \[
    \sigma
    >\;
    \Bigl(\tfrac{(\theta-1)(1-\chi_k)\varphi}{\lambda_k}
      - 1 - \varphi(1-\chi_k)\Bigr)\tfrac{1}{\omega_{c,k}}
    \;+\;1,
  \]
  Which is always true under flexible prices, and can be true under low price rigidities ($\alpha < 0.3$). More in general, $\kappa \neq 1$ implies a less countercyclical stance.
\end{frame}


\begin{frame}{Optimal Monetary Rule}
\begin{itemize}
\item Target weighted inflation index:
\[
\sum_k\frac{\theta(1-\chi_k)}{\lambda_k}\,\pi_{k,t} = -\sum_k \Psi_k(y_{k,t},\, g_{k,t},\, a_{k,t})
\]
\item Weights increase with private‑consumption share, and decrease with Phillips slope $\lambda_k$.
\item $\kappa>0$ lowers inflation volatility; high $\sigma$ raises weight on government spending.
\end{itemize}
\end{frame}

\section{Constrained Fiscal Capacity}
\begin{frame}{Fixed Government Basket}
\begin{itemize}
\item We set $\sigma = 1$, and $\kappa=0$ to abstract from other channels of sectoral interaction.
\item The government is required to deliver an exogenous CES bundle $\bar G_t$ (with elasticity $\rho$).
\item Optimal rule becomes \emph{relative}: choose residual sector $i$, set others as functions of gaps wrt $i$.
\item Preserves counter‑cyclical stance: allocate more spending to sectors with lower $\pi_{k,t}$ and $y_{k,t}$.
\end{itemize}
\end{frame}

\begin{frame}{Optimal policy}
\begin{itemize}
    \item \textbf{Fiscal policy} sets $g_{k,t}$ inversely proportional to the inflation and output gap between sector $k$ and the residual sector $i$.\\
    $g_{i,t}$ is a function of sectoral output and inflation in all sectors of the economy.
    \item \textbf{Monetary policy} is now reduced to a more general case where the optimal inflation target depends on sectoral output and government spending.
\end{itemize}
\end{frame}

\section{Conclusions}
\begin{frame}{Conclusions}
\begin{itemize}
\item The assumption that government prices are independent of consumer prices leads to a strong simplification of the optimal policy rule, and overestimates optimal policy countercyclicality. 
\item Constrained budgets require relative‐allocation rules rather than absolute spending paths.
\end{itemize}
\end{frame}


\begin{frame}{Thank You}
\centering Questions?
\end{frame}




\section*{Appendix}
\inappendixtrue
\begin{frame}[noframenumbering]
  \frametitle{Optimal Policy with $\kappa \neq 1$}
\small
Under non‐zero government‐demand pass‐through and CRRA preferences, the optimal fiscal rule is:
\[
  g_{k,t}
  = \frac{H_{k,t}}{X_{k,t}}\,y_{k,t}
  + \frac{J_{k,t}}{X_{k,t}}\,a_{k,t}
  - \frac{\theta}{\lambda_k\,X_{k,t}}\,\pi_{k,t}
  + \frac{\sigma - 1}{(1-\chi)^{1/\sigma}\,X_{k,t}}
    \sum_k \mu_k\lambda_k\,\phi_{k,t}^\pi
\]
\begingroup
\scriptsize
\[
\begin{aligned}
  H_{k,t} &= 
    \chi^{*}_{k}
  + 1 + \varphi
  + {\color{red}{(\sigma-1)\,\frac{\omega_{c,k}}{1-\chi_{k}}}}
  - \varphi\,\lambda_k\,
    \frac{1+\chi_{k}^* + \varphi + \tfrac{1}{\lambda_k\,(1-\chi_k)}}
         {\lambda_{k}\,\varphi
          + {\color{red}{\tfrac{\kappa}{\theta-1}\,\tfrac{\lambda_{k}}{1-\chi_{k}}}}},
  \\[1ex]
  J_{k,t} &= 
    -\,(1+\varphi)\,
      \Biggl(1
      - \lambda_k\,
        \frac{1+\chi_{k}^* + \varphi + \tfrac{1}{\lambda_k\,(1-\chi_k)}}
             {\lambda_{k}\,\varphi
              + {\color{red}{\tfrac{\kappa}{\theta-1}\,\tfrac{\lambda_{k}}{1-\chi_{k}}}}}
      \Biggr),
  \\[1ex]
  X_{k,t} &=
    \chi^{*}_{k}
  + {\color{red}{(\sigma-1)\,\chi^{*}_{k}\,\omega_{c,k}\,\chi_k}}
  + \bigl(1 + {\color{red}{(\sigma-1)\,\omega_{g,k}}}\bigr)\,
    \Biggl(
      \lambda_k
      + \frac{1+\chi_{k}^* + \varphi}
             {\lambda_{k}\,\varphi
              + {\color{red}{\tfrac{\kappa}{\theta-1}\,\tfrac{\lambda_{k}}{1-\chi_{k}}}}}
    \Biggr),
  \\[1ex]
  \phi_{k,t}^\pi &=
    \Bigl(-\,\varphi\,y_{k,t}
      - g_{k,t}\,(1+{\color{red}{(\sigma-1)\,\omega_{g,k}}})
      + (1+\varphi)\,a_{k,t}\Bigr)
    \Bigl(\lambda_{k}\,\varphi
      + {\color{red}{\tfrac{\kappa}{\theta-1}\,\tfrac{\lambda_{k}}{1-\chi_k}}}
    \Bigr)^{-1}.
\end{aligned}
\]
\endgroup
\end{frame}


\begin{frame}[noframenumbering]
  \frametitle{Optimal Policy with $\kappa \neq 1$}
  \small
Optimal monetary policy entails setting the aggregate inflation target:
  \begingroup
\begin{gather*}
    \sum_k \, \frac{\theta(1-\chi_k)}{\lambda_k} \, \pi_{k,t}\;=\; \notag \\[-4pt]
    \qquad 
    - \,\sum_k \, \mu_k\, 
\Bigl(
  y_{k,t}\,\varphi
  \;+\;g_{k,t}\Bigl(1+{\color{red}{(\sigma-1)}}\,\omega_{g,k}\Bigr)
  \;+\;(1+\varphi)\,a_{k,t}
\Bigr)\, \times \\[-4pt]
\Bigl(
  \lambda_{k}\,\varphi
  \;+\;{\color{red}{\tfrac{\kappa}{\theta-1}\,\tfrac{\lambda_{k}}{1-\chi_k}}}
\Bigr)^{-1}.
\end{gather*}
\endgroup
\end{frame}

\begin{frame}[noframenumbering]
\frametitle{Optimal Policy with Aggregate Constraint}
\small
Optimal provision under an aggregate constraint involves:
\begin{itemize}
  \item Selecting a residual sector $i$, whose government spending is freely set;
  \item Expressing $g_{k,t}$ in all other sectors $k \neq i$ as a function of relative output and inflation deviations with respect to sector $i$.
\end{itemize}

\medskip
The optimal spending rule becomes:
\begingroup
\scriptsize
\begin{align*}
g_{k,t} \;&=\; \tfrac{1-\chi_k}{1-\chi_i}\,\biggl(\tfrac{1+\lambda_i +\varphi \lambda_i}{1+\lambda_i + \phi\lambda_i(1-\chi_i)}\biggr)\, \biggl(\tfrac{\omega_{g,k}}{\omega_{g,i}}\biggr)^\rho\, \biggl(\tfrac{1+\lambda_k + \varphi \lambda_k}{1+\lambda_k + \phi\lambda_k(1-\chi_k)}\biggr)^{-1}\, g_{i,t}\\
&\quad -\; \tfrac{\varphi\, y_{k,t}}{1+\lambda_k + \varphi \lambda_k}
\;-\; \tfrac{\theta\, \varphi\, (1-\chi_k)\,\pi_{k,t}}{1+\lambda_k + \varphi \lambda_k}\\
&\quad +\; \tfrac{1-\chi_k}{1-\chi_i}
\biggl(\tfrac{\omega_{g,k}}{\omega_{g,i}}\biggr)^\rho 
\biggl(
\tfrac{1+\lambda_k + \varphi \lambda_k}{1+\lambda_k + \phi\lambda_k(1-\chi_k)}\biggr)^{-1}\\
&\qquad \times \biggl(
\tfrac{\varphi}{1+\lambda_i+\phi\lambda_i(1-\chi_i)}\,y_{i,t}
+ \tfrac{\theta\, \varphi\, (1-\chi_i)}{1+\lambda_i+\phi\lambda_i(1-\chi_i)}\,\pi_{i,t}
\biggr)
\end{align*}
\endgroup
\end{frame}

\begin{frame}[noframenumbering]
  \frametitle{Optimal Policy with $\kappa \neq 1$}
  \small
Optimal monetary policy entails setting the aggregate inflation target:
  \begingroup

\begin{gather}
    \sum_k \, \mu_k\, \frac{\theta \, (1-\chi_k)}{\lambda_k}\, \frac{\lambda_k + \varphi \,\lambda_k\,(1-\chi_k)}{1 + \lambda_k + \varphi \,\lambda_k\,(1-\chi_k)}\, \pi_{k,t} 
    \;=\; \notag \\[-4pt] \qquad
    \sum_k \, \mu_k \left( \frac{\chi_k\,g_{k,t}}{1 + \lambda_k + \varphi\,\lambda_k\,(1-\chi_k)} - \frac{y_{k,t}\,(1-\chi_k)\,(1+\varphi+\chi_k^*)}{1 + \lambda_k + \varphi \,\lambda_k\,(1-\chi_k)}\right). \notag
\end{gather}
\endgroup
\end{frame}












\end{document}