\documentclass[10pt,aspectratio=169]{beamer}

% ============================================================
% SHARED PREAMBLE
% ============================================================
% ==============================================================================
% SHARED PREAMBLE: Macro Workshop Presentations
% ==============================================================================
% University of Oxford, Department of Economics
% Use: % ==============================================================================
% SHARED PREAMBLE: Macro Workshop Presentations
% ==============================================================================
% University of Oxford, Department of Economics
% Use: % ==============================================================================
% SHARED PREAMBLE: Macro Workshop Presentations
% ==============================================================================
% University of Oxford, Department of Economics
% Use: \input{../Preambles/header.tex} from Slides/ directory
% ==============================================================================

% ============================================================
% THEME & COLOURS
% ============================================================
\usetheme{default}
\useinnertheme{rectangles}

\definecolor{oxfordblue}{RGB}{0,33,71}
\definecolor{oxfordmid}{RGB}{75,100,130}
\definecolor{oxfordlight}{RGB}{195,210,225}
\definecolor{oxfordaccent}{RGB}{163,31,52}
\definecolor{oxfordgrey}{RGB}{100,100,100}

\setbeamercolor{structure}{fg=oxfordblue}
\setbeamercolor{frametitle}{fg=white,bg=oxfordblue}
\setbeamercolor{title}{fg=white,bg=oxfordblue}
\setbeamercolor{block title}{fg=white,bg=oxfordblue}
\setbeamercolor{block body}{bg=oxfordlight!30}
\setbeamercolor{block title alerted}{fg=white,bg=oxfordaccent}
\setbeamercolor{block body alerted}{bg=oxfordaccent!8}
\setbeamercolor{alerted text}{fg=oxfordaccent}
\setbeamercolor{item}{fg=oxfordblue}
\setbeamercolor{subitem}{fg=oxfordmid}
\setbeamercolor{footline}{fg=oxfordgrey,bg=oxfordblue!8}

% ============================================================
% FONTS & LAYOUT
% ============================================================
\usepackage[utf8]{inputenc}
\usepackage[T1]{fontenc}
\usepackage{lmodern}
\usepackage{amsmath,amssymb}
\usepackage{mathtools}
\usepackage{microtype}
\usepackage{graphicx}
\usepackage{booktabs}
\usepackage{tikz}
\usetikzlibrary{arrows.meta,positioning,calc}
\usepackage{appendixnumberbeamer}
\hypersetup{colorlinks=true,linkcolor=oxfordblue,urlcolor=oxfordmid,citecolor=oxfordaccent}

\setbeamerfont{frametitle}{size=\large,series=\bfseries}
\setbeamerfont{title}{size=\Large,series=\bfseries}
\setbeamerfont{author}{size=\normalsize}
\setbeamerfont{institute}{size=\small}
\setbeamerfont{date}{size=\small}
\setbeamerfont{footnote}{size=\tiny}

\setbeamertemplate{itemize item}{\small\raisebox{0.12ex}{\color{oxfordblue}$\blacktriangleright$}}
\setbeamertemplate{itemize subitem}{\scriptsize\raisebox{0.1ex}{\color{oxfordmid}$\bullet$}}
\setbeamertemplate{navigation symbols}{}

\setbeamertemplate{frametitle}{%
  \nointerlineskip
  \begin{beamercolorbox}[wd=\paperwidth,ht=2.8ex,dp=1.2ex,leftskip=0.8em]{frametitle}%
    \usebeamerfont{frametitle}\insertframetitle%
  \end{beamercolorbox}%
}

\setbeamersize{text margin left=1.2em,text margin right=1.2em}

% ============================================================
% CUSTOM COMMANDS
% ============================================================
\newcommand{\red}[1]{{\color{oxfordaccent}#1}}
 from Slides/ directory
% ==============================================================================

% ============================================================
% THEME & COLOURS
% ============================================================
\usetheme{default}
\useinnertheme{rectangles}

\definecolor{oxfordblue}{RGB}{0,33,71}
\definecolor{oxfordmid}{RGB}{75,100,130}
\definecolor{oxfordlight}{RGB}{195,210,225}
\definecolor{oxfordaccent}{RGB}{163,31,52}
\definecolor{oxfordgrey}{RGB}{100,100,100}

\setbeamercolor{structure}{fg=oxfordblue}
\setbeamercolor{frametitle}{fg=white,bg=oxfordblue}
\setbeamercolor{title}{fg=white,bg=oxfordblue}
\setbeamercolor{block title}{fg=white,bg=oxfordblue}
\setbeamercolor{block body}{bg=oxfordlight!30}
\setbeamercolor{block title alerted}{fg=white,bg=oxfordaccent}
\setbeamercolor{block body alerted}{bg=oxfordaccent!8}
\setbeamercolor{alerted text}{fg=oxfordaccent}
\setbeamercolor{item}{fg=oxfordblue}
\setbeamercolor{subitem}{fg=oxfordmid}
\setbeamercolor{footline}{fg=oxfordgrey,bg=oxfordblue!8}

% ============================================================
% FONTS & LAYOUT
% ============================================================
\usepackage[utf8]{inputenc}
\usepackage[T1]{fontenc}
\usepackage{lmodern}
\usepackage{amsmath,amssymb}
\usepackage{mathtools}
\usepackage{microtype}
\usepackage{graphicx}
\usepackage{booktabs}
\usepackage{tikz}
\usetikzlibrary{arrows.meta,positioning,calc}
\usepackage{appendixnumberbeamer}
\hypersetup{colorlinks=true,linkcolor=oxfordblue,urlcolor=oxfordmid,citecolor=oxfordaccent}

\setbeamerfont{frametitle}{size=\large,series=\bfseries}
\setbeamerfont{title}{size=\Large,series=\bfseries}
\setbeamerfont{author}{size=\normalsize}
\setbeamerfont{institute}{size=\small}
\setbeamerfont{date}{size=\small}
\setbeamerfont{footnote}{size=\tiny}

\setbeamertemplate{itemize item}{\small\raisebox{0.12ex}{\color{oxfordblue}$\blacktriangleright$}}
\setbeamertemplate{itemize subitem}{\scriptsize\raisebox{0.1ex}{\color{oxfordmid}$\bullet$}}
\setbeamertemplate{navigation symbols}{}

\setbeamertemplate{frametitle}{%
  \nointerlineskip
  \begin{beamercolorbox}[wd=\paperwidth,ht=2.8ex,dp=1.2ex,leftskip=0.8em]{frametitle}%
    \usebeamerfont{frametitle}\insertframetitle%
  \end{beamercolorbox}%
}

\setbeamersize{text margin left=1.2em,text margin right=1.2em}

% ============================================================
% CUSTOM COMMANDS
% ============================================================
\newcommand{\red}[1]{{\color{oxfordaccent}#1}}
 from Slides/ directory
% ==============================================================================

% ============================================================
% THEME & COLOURS
% ============================================================
\usetheme{default}
\useinnertheme{rectangles}

\definecolor{oxfordblue}{RGB}{0,33,71}
\definecolor{oxfordmid}{RGB}{75,100,130}
\definecolor{oxfordlight}{RGB}{195,210,225}
\definecolor{oxfordaccent}{RGB}{163,31,52}
\definecolor{oxfordgrey}{RGB}{100,100,100}

\setbeamercolor{structure}{fg=oxfordblue}
\setbeamercolor{frametitle}{fg=white,bg=oxfordblue}
\setbeamercolor{title}{fg=white,bg=oxfordblue}
\setbeamercolor{block title}{fg=white,bg=oxfordblue}
\setbeamercolor{block body}{bg=oxfordlight!30}
\setbeamercolor{block title alerted}{fg=white,bg=oxfordaccent}
\setbeamercolor{block body alerted}{bg=oxfordaccent!8}
\setbeamercolor{alerted text}{fg=oxfordaccent}
\setbeamercolor{item}{fg=oxfordblue}
\setbeamercolor{subitem}{fg=oxfordmid}
\setbeamercolor{footline}{fg=oxfordgrey,bg=oxfordblue!8}

% ============================================================
% FONTS & LAYOUT
% ============================================================
\usepackage[utf8]{inputenc}
\usepackage[T1]{fontenc}
\usepackage{lmodern}
\usepackage{amsmath,amssymb}
\usepackage{mathtools}
\usepackage{microtype}
\usepackage{graphicx}
\usepackage{booktabs}
\usepackage{tikz}
\usetikzlibrary{arrows.meta,positioning,calc}
\usepackage{appendixnumberbeamer}
\hypersetup{colorlinks=true,linkcolor=oxfordblue,urlcolor=oxfordmid,citecolor=oxfordaccent}

\setbeamerfont{frametitle}{size=\large,series=\bfseries}
\setbeamerfont{title}{size=\Large,series=\bfseries}
\setbeamerfont{author}{size=\normalsize}
\setbeamerfont{institute}{size=\small}
\setbeamerfont{date}{size=\small}
\setbeamerfont{footnote}{size=\tiny}

\setbeamertemplate{itemize item}{\small\raisebox{0.12ex}{\color{oxfordblue}$\blacktriangleright$}}
\setbeamertemplate{itemize subitem}{\scriptsize\raisebox{0.1ex}{\color{oxfordmid}$\bullet$}}
\setbeamertemplate{navigation symbols}{}

\setbeamertemplate{frametitle}{%
  \nointerlineskip
  \begin{beamercolorbox}[wd=\paperwidth,ht=2.8ex,dp=1.2ex,leftskip=0.8em]{frametitle}%
    \usebeamerfont{frametitle}\insertframetitle%
  \end{beamercolorbox}%
}

\setbeamersize{text margin left=1.2em,text margin right=1.2em}

% ============================================================
% CUSTOM COMMANDS
% ============================================================
\newcommand{\red}[1]{{\color{oxfordaccent}#1}}


% ============================================================
% PRESENTATION-SPECIFIC: FOOTLINE
% ============================================================
\setbeamertemplate{footline}{%
  \vskip2pt%
  \hbox{%
    \begin{beamercolorbox}[wd=\paperwidth,ht=2.8ex,dp=1.6ex,leftskip=1.2em,rightskip=1.2em]{footline}%
      {\usebeamerfont{footnote}\color{oxfordgrey}%
        A.\,Eugenelo%
        \hfill%
        Fiscal Policy in Production Networks%
        \hfill%
        \insertframenumber\,/\,\inserttotalframenumber%
      }%
    \end{beamercolorbox}%
  }%
}

% ============================================================
% METADATA
% ============================================================
\title[Fiscal Policy in Production Networks]{%
  Government Spending in Multi-Sector\\[3pt]
  Open Economies with Production Networks}
\author[A.\,Eugenelo]{Antonio Eugenelo}
\institute{University of Oxford\\ Department of Economics}
\date{Macro Workshop, February 2026}

% ============================================================
\begin{document}
% ============================================================

% ------ TITLE ------
{
\setbeamertemplate{footline}{}
\begin{frame}[plain]
  \vfill
  \begin{beamercolorbox}[wd=\paperwidth,ht=0.45\paperheight,dp=0pt,center]{title}
    \vskip1.5em
    \usebeamerfont{title}\inserttitle\\[12pt]
    \usebeamerfont{author}\insertauthor\\[4pt]
    \usebeamerfont{institute}\insertinstitute\\[8pt]
    \usebeamerfont{date}\insertdate
    \vskip1em
  \end{beamercolorbox}
  \vfill
\end{frame}
}

% ============================================================
% SLIDE 1 --- MOTIVATION
% ============================================================

\begin{frame}{Motivation}
  \textbf{Question.}  What is the welfare effect of redistributing sector-specific government spending under a fixed aggregate budget, relative to fully endogenous fiscal policy?

  \bigskip
  \begin{itemize}
    \setlength{\itemsep}{6pt}
    \item Production networks amplify and reshape the transmission of shocks across sectors (Acemoglu et al., 2012; Baqaee \& Farhi, 2020; Rubbo, 2023).
    \item Optimal fiscal allocation across sectors depends on sectoral heterogeneity in price stickiness and public-good shares (Cox et al., 2024), amplified by network position (Rubbo, 2023).
  \end{itemize}
\end{frame}

% ============================================================
% SLIDE 1b --- PLAN
% ============================================================

\begin{frame}{Plan}
  \begin{enumerate}
    \setlength{\itemsep}{12pt}
    \item[\textbf{(i)}] A \textbf{simple closed-economy model} with lump-sum transfers and government procurements $\to$ a \emph{relative allocation rule}.
    \item[\textbf{(ii)}] The \textbf{global production network model} of Aguilar et al.\ (2025): $K{=}4$ countries, $I{=}44$ sectors, with IO linkages and sectoral tariffs.
    \item[\textbf{(iii)}] \textbf{Original extensions}: time-varying labour income tax and production subsidy that enter the Phillips curves directly.
    \item[\textbf{(iv)}] \textbf{Research agenda}: introduce sector-specific government purchases in the networked model; quantify the welfare cost of fixing the aggregate budget.
  \end{enumerate}

  \medskip
  {\small\textit{Status:} steps (i)--(iii) are derived; the networked model implementation (iv) is in progress.}
\end{frame}

% ============================================================
% SLIDE 2 --- SIMPLE MODEL: SETUP + WELFARE
% ============================================================

\begin{frame}{A Simple Multi-Sector Model}
  Closed NK economy, $N$ sectors, Calvo pricing ($\alpha_k$), linear production $Y_{k,t} = A_{k,t}\,N_{k,t}$, following Cox et al.\ (2024).

  \smallskip
  \textbf{Key constraint:} the aggregate public-good bundle $\bar{G}_t$ is \textit{exogenous}.  The planner chooses only the sectoral composition $\{g_{k,t}\}$.

  \medskip
  Setting $\sigma=1$, the second-order welfare approximation is (where $\chi_k^* \equiv \chi_k/(1-\chi_k)$):
  \[
    \mathcal{W} \;\approx\; -\,\frac{1}{2}\,\sum_{k}\mu_k\bigg[\underbrace{(1+\varphi)\,y_{k,t}^2}_{\text{output gaps}} + \underbrace{\frac{\theta(1-\chi_k)}{\lambda_k}\,\pi_{k,t}^2}_{\text{inflation}} + \underbrace{\chi_k^*\,(g_{k,t}-y_{k,t})^2}_{\text{public-good gaps}}\bigg]
  \]

  \smallskip
  Three tensions the planner must balance:
  \begin{enumerate}
    \setlength{\itemsep}{2pt}
    \item \textbf{Output-gap stabilisation:} penalises $y_{k,t}^2$.
    \item \textbf{Inflation stabilisation:} penalises $\pi_{k,t}^2$, weighted inversely by the PC slope $\lambda_k$.
    \item \textbf{Public-good allocation:} penalises $g_{k,t}-y_{k,t}$; under a fixed budget the planner can only \textit{reshuffle} spending across sectors.
  \end{enumerate}
\end{frame}

% ============================================================
% SLIDE 3 --- THE RELATIVE ALLOCATION RULE
% ============================================================

\begin{frame}[label=allocation-rule]{The Relative Allocation Rule}
  Under exogenous $\bar{G}_t$, the spending gap between sector~$k$ and a residual sector~$i$ satisfies:\footnote{$\Phi_{ki}$ collects structural parameters: public-good shares $\chi_k,\chi_i$, aggregator weights $\omega_{g,k}/\omega_{g,i}$, and PC slopes $\lambda_k,\lambda_i$.  \hyperlink{structural-coefficients}{\beamergotobutton{Extended form}}}
  \begingroup\large
  \[
    \underbrace{g_{k,t} - \Phi_{ki}\,g_{i,t}}_{\text{\normalsize spending gap}}
    \;=\;
    -\;\underbrace{\bigl(a_k\,y_{k,t} - \Phi_{ki}\,a_i\,y_{i,t}\bigr)}_{\text{\normalsize\red{output-gap differential}}}
    \;-\;\underbrace{\bigl(b_k\,\pi_{k,t} - \Phi_{ki}\,b_i\,\pi_{i,t}\bigr)}_{\text{\normalsize\red{inflation differential}}}
  \]
  \endgroup
  {\small where $a_k \equiv \frac{\varphi}{1{+}\lambda_k{+}\varphi\lambda_k}$\,,\; $b_k \equiv \frac{\theta\,\varphi\,(1{-}\chi_k)}{1{+}\lambda_k{+}\varphi\lambda_k}$\,,\; and analogously for sector~$i$.}

  \bigskip
  \begin{alertblock}{Key property}
    The rule is inherently \textit{relative}: spending is \textbf{reallocated} toward sectors with lower inflation and lower output gaps \emph{relative} to the residual sector.  What matters is the cross-sectional differences in gaps, not their level.
  \end{alertblock}
\end{frame}

% ============================================================
% SLIDE 4 --- AGUILAR ET AL. MODEL
% ============================================================

\begin{frame}[label=production-network]{The Global Production Network Model (Aguilar et al., 2025)}
  The closed-economy model delivers the intuition; we need a quantitative framework to test it.

  \medskip
  \begin{columns}[T]
    \begin{column}{0.48\textwidth}
      \begin{itemize}
        \setlength{\itemsep}{4pt}
        \item $K{=}4$ countries, $I{=}44$ sectors
        \item Nested CES: energy/non-energy, domestic/foreign
        \item Sector- and country-specific Calvo pricing
        \item Country-specific Taylor rules
        \item Balanced budget, lump-sum taxes, static production subsidies, tariff revenue
      \end{itemize}

      \medskip
      \centering
      \begin{tikzpicture}[
        node distance=1.2cm and 1.4cm,
        sector/.style={circle, draw=oxfordblue, fill=oxfordblue!10, thick,
                       minimum size=0.8cm, font=\footnotesize},
        >=Stealth
      ]
        \node[sector] (s1) {$i$};
        \node[sector, below right=0.8cm and 1.2cm of s1] (s2) {$j$};
        \node[sector, below left=0.8cm and 1.2cm of s1] (s3) {$l$};
        % IO links
        \draw[->, thick, oxfordblue] (s1) -- node[right, font=\scriptsize] {$\omega_{kij}$} (s2);
        \draw[->, thick, oxfordblue] (s2) -- node[below, font=\scriptsize] {$\omega_{kjl}$} (s3);
        \draw[->, thick, oxfordblue] (s3) -- node[left, font=\scriptsize] {$\omega_{kli}$} (s1);
        % Government spending
        \draw[->, thick, oxfordaccent] ([xshift=-0.6cm, yshift=0.6cm]s3.north west) -- node[left, font=\scriptsize, text=oxfordaccent] {$G_{kl}$} (s3);
      \end{tikzpicture}

      {\scriptsize IO network with fiscal entry point}
    \end{column}
    \begin{column}{0.48\textwidth}
      Marginal cost in sector $i$, country $k$:
      \[
        \widehat{\mathrm{mc}}_{ki,t} = -a_{ki,t} + \underbrace{\mathcal{M}_{ki}\alpha_{ki}\,\widehat{w}_{k,t}}_{\text{labour}} + \underbrace{\textstyle\sum_{l,j} \mathcal{M}_{ki}\omega_{klij}\,\widehat{p}_{klij,t}}_{\text{IO inputs}}
      \]
      Sectoral Phillips curve:
      \[
        \pi_{ki,t} = \kappa_{ki}\!\left(\widehat{\mathrm{mc}}_{ki,t} - \widehat{p}_{ki,t}\right) + \beta\,\mathbb{E}_t\pi_{ki,t+1} + u^p_{ki,t}
      \]
      Tariffs enter as price wedges (\hyperlink{tariff-propagation}{\beamergotobutton{Details}}).
    \end{column}
  \end{columns}
\end{frame}

% ============================================================
% SLIDE 5 --- EXTENSIONS: FISCAL INSTRUMENTS
% ============================================================

\begin{frame}{Our Contribution: Fiscal Instruments in the Phillips Curves}
  \textbf{Why extend?}  The Aguilar et al.\ model features lump-sum taxes and static production subsidies but no active fiscal stabilisation.  We introduce two time-varying distortionary instruments that enter the Phillips curves directly.  This is a tractable first step; the full Ramsey problem in a $K \times I$ networked model is the research goal.

  \medskip
  Define $\hat{\tau}^w_{k,t} \equiv (\tau^w_{k,t} - \bar{\tau}^w_k)/(1-\bar{\tau}^w_k)$ and $\hat{\tau}^s_{ki,t} \equiv (\tau^s_{ki,t} - \bar{\tau}^s_{ki})/(1-\bar{\tau}^s_{ki})$.

  \bigskip
  \textbf{Wage Phillips curve,} adding a labour income tax:
  \[
    \pi_{wk,t} = \kappa_{wk}\Big(\sigma\,\hat{c}_{k,t} + \varphi\,\hat{n}_{k,t} - \hat{w}_{k,t} + \red{\hat{\tau}^w_{k,t}}\Big) + \beta\,\mathbb{E}_t\pi_{wk,t+1} + u^w_{k,t}
  \]

  \textbf{Price Phillips curve,} making the production subsidy time-varying:
  \[
    \pi_{ki,t} = \kappa_{ki}\Big(\widehat{\mathrm{mc}}_{ki,t} - \hat{p}_{ki,t} \;\red{-\; \hat{\tau}^s_{ki,t}}\Big) + \beta\,\mathbb{E}_t\pi_{ki,t+1} + u^p_{ki,t}
  \]

  \begin{alertblock}{Key implication}
    Tax hike $\to$ inflationary cost-push.\quad Subsidy hike $\to$ disinflationary cost-push.
  \end{alertblock}
\end{frame}

% ============================================================
% SLIDE 6 --- RESEARCH AGENDA
% ============================================================

\begin{frame}{Research Agenda: Fixed vs.\ Endogenous Fiscal Envelopes}
  \textbf{Goal:} introduce sector-specific $G_{ki,t}$ in the goods market clearing condition of the global production network model:
  \[
    Y_{ki,t} = \sum_l C_{lki,t} + \sum_l\sum_j X_{lkji,t} + \red{G_{ki,t}}
  \]
  and compare two policy regimes:

  \medskip
  \begin{columns}[T]
    \begin{column}{0.47\textwidth}
      \begin{block}{Fully endogenous spending}
        Planner chooses level \textit{and} composition.  Unconstrained fiscal benchmark (first-best within the class of spending instruments).
      \end{block}
    \end{column}
    \begin{column}{0.47\textwidth}
      \begin{block}{Fixed aggregate budget}
        $\bar{G}_{k,t}$ exogenous; only the sectoral composition adjusts.  Relevant when total spending is politically constrained.
      \end{block}
    \end{column}
  \end{columns}

  \bigskip
  \textbf{Central question:} how large is the welfare gap between the two regimes?  If it is small, compositional reallocation alone (the relative allocation rule) may approximate the first-best, even without aggregate fiscal flexibility.

\end{frame}

% ============================================================
% SLIDE 7 --- SUMMARY
% ============================================================

\begin{frame}{Summary}
  \begin{enumerate}
    \setlength{\itemsep}{12pt}
    \item \textbf{Key result:} under a fixed fiscal envelope, optimal spending follows a \emph{relative allocation rule}; what matters is the cross-sectional dispersion of output and inflation gaps, not their level.
    \item \textbf{Central question:} can compositional reallocation alone approximate the first-best, even without aggregate fiscal flexibility?
    \item \textbf{What's next:} quantify the welfare gap between fixed and endogenous budgets in the $K \times I$ networked model of Aguilar et al.\ (2025), extended with distortionary fiscal instruments.
  \end{enumerate}
\end{frame}

% ============================================================
% THANK YOU
% ============================================================
{
\setbeamertemplate{footline}{}
\begin{frame}[plain]
  \vfill
  \begin{center}
    {\Large\color{oxfordblue}\textbf{Thank you}}
  \end{center}
  \vfill
\end{frame}
}

% ============================================================
%
%                        APPENDIX
%
% ============================================================
\appendix

% ------ A1: FULL LITERATURE ------

\begin{frame}[noframenumbering]{Appendix: Related Literature}
  \begin{columns}[T]
    \begin{column}{0.48\textwidth}
      \begin{block}{Production Networks \& NK}
        \begin{itemize}
          \item Acemoglu et al.\ (2012)
          \item Baqaee \& Farhi (2020, 2024)
          \item Pasten, Schoenle \& Weber (2020)
          \item Rubbo (2023)
        \end{itemize}
      \end{block}
      \vspace{4pt}
      \begin{block}{Tariffs \& Open-Economy NK}
        \begin{itemize}
          \item Gal\'{i} \& Monacelli (2005)
          \item Comin \& Johnson (2023)
          \item Aguilar et al.\ (2025)
        \end{itemize}
      \end{block}
    \end{column}
    \begin{column}{0.48\textwidth}
      \begin{block}{Fiscal Policy in Disaggregated Economies}
        \begin{itemize}
          \item Aoki (2001)
          \item Antonova \& M\"{u}ller (2025)
          \item Cox, Feng, M\"{u}ller, Pasten, Schoenle \& Weber (2024)
        \end{itemize}
      \end{block}
      \vspace{4pt}
      \begin{block}{Fiscal--Price Effects (Empirical)}
        \begin{itemize}
          \item Nekarda \& Ramey (2020)
          \item Ben Zeev \& Pappa (2017)
        \end{itemize}
      \end{block}
    \end{column}
  \end{columns}
\end{frame}

% ------ A3: FULL WELFARE WITH SIGMA =/= 1, KAPPA =/= 0 ------

\begin{frame}[noframenumbering]{Appendix: Welfare with $\sigma \neq 1$ and $\kappa \neq 0$}
  When CRRA preferences and government-demand pass-through are active:
  \small
  \begin{align*}
    -\frac{1}{2}\sum_k\mu_k\Biggl(&
      (1{+}\varphi)\,y_{k,t}^2
      + \frac{\theta(1{-}\chi_k)}{\lambda_k}\,\pi_{k,t}^2
      + \chi_k^*\,(g_{k,t}{-}y_{k,t})^2 \\
    & + \red{(\sigma{-}1)}\Biggl[
        (1{-}\chi_k)\,\omega_{c,k}\!\left(\frac{y_{k,t}}{1{-}\chi_k} - \chi_k^*\,g_{k,t}\right)^{\!2}
        + \omega_{g,k}\,\chi_k\,g_{k,t}^2
      \Biggr]\Biggr)
  \end{align*}

  \begin{itemize}
    \item The $\sigma{-}1$ term introduces an \textbf{insurance motive}: the planner uses sectoral fiscal policy to hedge against aggregate risk.
    \item $\kappa > 0$ steepens Phillips curves ($\lambda'_k > \lambda_k$), making fiscal policy a supply-side instrument.
  \end{itemize}
\end{frame}

% ------ A4: RELATIVE RULE (structural coefficients) ------

\begin{frame}[noframenumbering,label=structural-coefficients]{Appendix: Relative Allocation Rule --- Structural Coefficients}
  \small
  Under exogenous $\bar{G}_t$ (with $\sigma=1$, $\kappa=0$):
  \begingroup\footnotesize
  \begin{align*}
    g_{k,t} \;&=\; \tfrac{1{-}\chi_k}{1{-}\chi_i}\biggl(\tfrac{1{+}\lambda_i{+}\varphi\lambda_i}{1{+}\lambda_i{+}\varphi\lambda_i(1{-}\chi_i)}\biggr) \biggl(\tfrac{\omega_{g,k}}{\omega_{g,i}}\biggr)^{\!\rho} \biggl(\tfrac{1{+}\lambda_k{+}\varphi\lambda_k}{1{+}\lambda_k{+}\varphi\lambda_k(1{-}\chi_k)}\biggr)^{\!-1} g_{i,t} \\
    &\quad -\; \tfrac{\varphi\,y_{k,t}}{1{+}\lambda_k{+}\varphi\lambda_k}
    \;-\; \tfrac{\theta\,\varphi\,(1{-}\chi_k)\,\pi_{k,t}}{1{+}\lambda_k{+}\varphi\lambda_k} \\
    &\quad +\; \tfrac{1{-}\chi_k}{1{-}\chi_i}\biggl(\tfrac{\omega_{g,k}}{\omega_{g,i}}\biggr)^{\!\rho}
    \biggl(\tfrac{1{+}\lambda_k{+}\varphi\lambda_k}{1{+}\lambda_k{+}\varphi\lambda_k(1{-}\chi_k)}\biggr)^{\!-1} \\
    &\qquad \times\biggl(
      \tfrac{\varphi\,y_{i,t}}{1{+}\lambda_i{+}\varphi\lambda_i(1{-}\chi_i)}
      + \tfrac{\theta\,\varphi\,(1{-}\chi_i)\,\pi_{i,t}}{1{+}\lambda_i{+}\varphi\lambda_i(1{-}\chi_i)}
    \biggr)
  \end{align*}
  \endgroup

  $\omega_{g,k}$: weight of sector $k$ in the government Cobb--Douglas aggregator.\quad $\rho$: CES elasticity of public-good bundle.

  \smallskip
  {\scriptsize \textit{Note:} the asymmetric denominators ($1{+}\lambda_k{+}\varphi\lambda_k$ for own-sector terms vs.\ $1{+}\lambda_i{+}\varphi\lambda_i(1{-}\chi_i)$ for the residual-sector terms) arise from solving the equated FOCs for $g_{k,t}$; the $(1{-}\chi)$ factor enters through the resource-constraint substitution in the residual sector's Phillips curve.}

  \vfill
  \hfill\hyperlink{allocation-rule}{\beamerreturnbutton{Back}}
\end{frame}

% ------ A5: OPTIMAL MONETARY RULE UNDER CONSTRAINT ------

\begin{frame}[noframenumbering]{Appendix: Optimal Monetary Rule under Aggregate Constraint}
  Under exogenous $\bar{G}_t$, optimal monetary policy sets:
  \begin{gather*}
    \sum_k \mu_k\,\frac{\theta(1{-}\chi_k)}{\lambda_k}\,\frac{\lambda_k + \varphi\lambda_k(1{-}\chi_k)}{1+\lambda_k+\varphi\lambda_k(1{-}\chi_k)}\,\pi_{k,t}
    \;= \\[4pt]
    \sum_k \mu_k\!\left(\frac{\chi_k\,g_{k,t}}{1+\lambda_k+\varphi\lambda_k(1{-}\chi_k)} - \frac{y_{k,t}(1{-}\chi_k)(1{+}\varphi{+}\chi_k^*)}{1+\lambda_k+\varphi\lambda_k(1{-}\chi_k)}\right)
  \end{gather*}

  \begin{itemize}
    \item Inflation weights depend on private-consumption share and $\lambda_k$.
    \item Government spending enters the target because the constraint links $g_{k,t}$ and $y_{k,t}$ across sectors.
  \end{itemize}
\end{frame}

% ------ A6: AGUILAR ET AL. MODEL DETAILS ------

\begin{frame}[noframenumbering]{Appendix: Aguilar et al.\ ---Household Problem}
  \small
  Per-period utility: $U_t = \bigl(C_{k,t}^{1-\sigma}/(1{-}\sigma) - \int_0^1 \mathcal{N}_{gk,t}^{1+\varphi}/(1{+}\varphi)\,dg\bigr)Z_{k,t}$

  \medskip
  Consumption nested CES (energy/non-energy, domestic/foreign):
  \[
    C_{k,t} = \left[\widetilde{\beta}_k^{1/\gamma}\,C_{kE,t}^{(\gamma-1)/\gamma} + (1{-}\widetilde{\beta}_k)^{1/\gamma}\,C_{kM,t}^{(\gamma-1)/\gamma}\right]^{\gamma/(\gamma-1)}
  \]

  Euler equations:
  \begin{align*}
    C_{k,t}^{-\sigma} &= \beta\,\mathbb{E}_t\,C_{k,t+1}^{-\sigma}\,\frac{1+i_{k,t}}{1+\pi_{kC,t+1}}\,\frac{Z_{k,t+1}}{Z_{k,t}} \\[2pt]
    i_{k,t} - i_{K,t} &= \mathbb{E}_t\Delta e_{kK,t+1} - \gamma_*\,\mathrm{nfa}_{k,t} + \varepsilon^e_{kK,t} \qquad\text{(UIP)}
  \end{align*}

  Calvo wage setting yields:
  \[
    \pi_{wk,t} = \kappa_{wk}\bigl(\sigma\hat{c}_{k,t} + \varphi\hat{n}_{k,t} - \hat{w}_{k,t}\bigr) + \beta\,\mathbb{E}_t\pi_{wk,t+1} + u^w_{k,t}
  \]
\end{frame}

\begin{frame}[noframenumbering]{Appendix: Aguilar et al.\ ---Firm Problem}
  \small
  CES production: $Y_{ki,f,t} = A_{ki,t}\!\left[\widetilde{\alpha}_{ki}^{1/\psi}\,N_{fki,t}^{(\psi-1)/\psi} + \widetilde{\vartheta}_{ki}^{1/\psi}\,X_{fki,t}^{(\psi-1)/\psi}\right]^{\psi/(\psi-1)}$

  \medskip
  Intermediate bundle mirrors the household CES nesting (energy/non-energy, domestic/foreign).

  \medskip
  Log-linearised marginal cost:
  \[
    \widehat{\mathrm{mc}}_{ki,t} = -a_{ki,t} + \mathcal{M}_{ki}\alpha_{ki}\,\widehat{w}_{k,t} + \sum_{l=1}^K\sum_{j=1}^I \mathcal{M}_{ki}\omega_{klij}\,\widehat{p}_{klij,t}
  \]
  where $\alpha_{ki}$: labour share, $\omega_{klij}$: IO expenditure share, $\mathcal{M}_{ki}$: steady-state markup.

  \medskip
  Calvo pricing yields:
  \[
    \pi_{ki,t} = \kappa_{ki}\bigl(\widehat{\mathrm{mc}}_{ki,t} - \widehat{p}_{ki,t}\bigr) + \beta\,\mathbb{E}_t\pi_{ki,t+1} + u^p_{ki,t}
    \qquad
    \kappa_{ki} = \frac{(1-\theta^p_{ki})(1-\beta\theta^p_{ki})}{\theta^p_{ki}}
  \]
\end{frame}

\begin{frame}[noframenumbering,label=tariff-propagation]{Appendix: Aguilar et al.\ ---Tariff Propagation}
  \small
  Tariffs enter as price wedges on final and intermediate goods:
  \[
    P_{k,l,i,t} = (1+\tau_{k,l,i,t})\,\widetilde{P}_{l,k,i,t}
  \]

  \textbf{Propagation:}
  \begin{enumerate}
    \setlength{\itemsep}{4pt}
    \item Tariff on country~$l$ raises input prices $\widehat{p}_{klij,t}$ for domestic sectors sourcing from sector~$j$ in~$l$.
    \item Higher input costs raise $\widehat{\mathrm{mc}}_{ki,t}$, feeding into $\pi_{ki,t}$.
    \item Cost increases cascade downstream through the IO network.
    \item Tariff revenue accrues to the government; currently rebated lump-sum.
  \end{enumerate}

  \medskip
  Government budget constraint:
  \[
    \frac{B_{k,t}}{1+i_{k,t}} + T_{k,t} + \sum_{l\neq k}\sum_i \tau_{kli,t}\,P^l_{kli,t}\!\left(C_{kli,t} + \textstyle\sum_j X_{klji,t}\right) = B_{k,t-1} + \sum_i \tau^s_{ki}\,\mathrm{MC}_{ki,t}\,Y_{ki,t}
  \]

  \vfill
  \hfill\hyperlink{production-network}{\beamerreturnbutton{Back}}
\end{frame}

\begin{frame}[noframenumbering]{Appendix: Aguilar et al.\ ---Calibration}
  \small
  \begin{columns}[T]
    \begin{column}{0.48\textwidth}
      \textbf{Households}
      \begin{itemize}
        \item $\beta=0.99$, $\sigma=1$, $\varphi=1$
        \item Energy/non-energy elast.\ $\gamma=0.4$
        \item Trade elasticity $\delta=1$
        \item Calvo wage $\theta^w_k=0.75$
        \item Consumption shares from OECD ICIO (2019)
      \end{itemize}

      \medskip
      \textbf{Monetary policy}
      \begin{itemize}
        \item $\rho_r=0.7$, $\phi_\pi=1.5$, $\phi_y=0.125$
        \item Target: headline inflation
      \end{itemize}
    \end{column}
    \begin{column}{0.48\textwidth}
      \textbf{Firms}
      \begin{itemize}
        \item Labour/input elast.\ $\psi=0.5$
        \item Energy/non-energy elast.\ $\phi=0.4$
        \item Trade elasticity $\mu=1$
        \item IO shares from OECD ICIO (2019)
        \item Markups from Eurostat Figaro
        \item Calvo prices from ECB PRISMA
      \end{itemize}

      \medskip
      \textbf{Tariff shocks}
      \begin{itemize}
        \item $\rho^\tau=0.96$, $\sigma^\tau=1$
      \end{itemize}
    \end{column}
  \end{columns}
\end{frame}

\begin{frame}[noframenumbering]{Appendix: Aguilar et al.\ ---Goods Market Clearing and GDP}
  \small
  Market clearing:
  \[
    Y_{ki,t} = \sum_{l=1}^K C_{lki,t} + \sum_{l=1}^K\sum_{j=1}^I X_{lkji,t}
  \]

  Log-linearised:
  \[
    \lambda_{ki}\,\widehat{y}_{ki,t} = \sum_{l=1}^K \mathcal{Y}_{lk}\!\left(\beta_{lki}\,\widehat{c}_{lki,t} + \sum_{j=1}^I \lambda_{lj}\,\omega_{lkji}\,\widehat{x}_{lkji,t}\right)
  \]
  where $\lambda_{ki} = P_{ki}Y_{ki}/\mathcal{Y}_k$ is the Domar weight.

  \medskip
  Real GDP:
  \[
    \widehat{y}_{k,t} = \widehat{c}_{k,t} + \Upsilon_k\!\left(\widehat{\mathrm{exp}}_{k,t} - \widehat{\mathrm{imp}}_{k,t}\right)
  \]
  where $\Upsilon_k$ is the trade-to-GDP ratio.
\end{frame}

% ------ A7: EXTENSION DERIVATIONS ------

\begin{frame}[noframenumbering]{Appendix: Derivation ---Wage Phillips Curve with Tax}
  \small
  Household FOC with labour income tax $\tau^w_{k,t}$:
  \[
    \sum_{l=0}^{\infty}(\beta\theta^W_k)^l\,\mathbb{E}_t\!\left[N_{k,t+l|t}\,C_{t+l|t}^{-\sigma}\!\left(\frac{(1{-}\tau^w_{k,t+l})\,W^*_{k,t}}{P_{t+l}} - \mathcal{M}_{wk,t}\,\mathrm{MRS}_{k,t+l|t}\right)\right] = 0
  \]

  Log-linearised reset wage:
  \[
    w^*_{k,t} = (1{-}\beta\theta^W_k)\sum_{l=0}^{\infty}(\beta\theta^W_k)^l\,\mathbb{E}_t\!\left[\mathrm{mrs}_{t+l|t} + \mu^n_{wk,t+l} + p_{kC,t+l} + \hat{\tau}^w_{k,t+l}\right]
  \]

  Calvo aggregation ($\pi_{wk,t} = w_{k,t} - w_{k,t-1}$):
  \[
    \pi_{wk,t} = \kappa_{wk}\bigl(\sigma\hat{c}_{k,t} + \varphi\hat{n}_{k,t} - \hat{w}_{k,t} + \hat{\tau}^w_{k,t}\bigr) + \beta\,\mathbb{E}_t\pi_{wk,t+1} + u^w_{k,t}
  \]
  Tax deviation enters additively: wage-setters pass the wedge through to the pre-tax wage.
\end{frame}

\begin{frame}[noframenumbering]{Appendix: Derivation ---Price Phillips Curve with Subsidy}
  \small
  Firm FOC with time-varying production subsidy $\tau^s_{ki,t}$:
  \[
    \sum_{l=0}^{\infty}(\beta\theta^p_{ki})^l\,\mathbb{E}_t\!\left[\Lambda_{t,t+l}\,Y_{ki,t+l|t}\!\left(P^*_{ki,t} - \mathcal{M}_{pk,t+l}\,(1{-}\tau^s_{ki,t+l})\,\mathrm{MC}^n_{ki,t+l|t}\right)\right] = 0
  \]

  Log-linearised reset price:
  \[
    p^*_{ki,t} = (1{-}\beta\theta^p_{ki})\sum_{l=0}^{\infty}(\beta\theta^p_{ki})^l\,\mathbb{E}_t\!\left[\mathrm{mc}^n_{ki,t+l|t} + \mu^n_{pki,t+l} - \hat{\tau}^s_{ki,t+l}\right]
  \]

  Calvo aggregation ($\pi_{ki,t} = p_{ki,t} - p_{ki,t-1}$):
  \[
    \pi_{ki,t} = \kappa_{ki}\bigl(\widehat{\mathrm{mc}}_{ki,t} - \hat{p}_{ki,t} - \hat{\tau}^s_{ki,t}\bigr) + \beta\,\mathbb{E}_t\pi_{ki,t+1} + u^p_{ki,t}
  \]
  Subsidy increase lowers effective marginal cost $\to$ disinflationary cost-push.
\end{frame}

% ------ A8: BUDGET DYNAMICS ------

\begin{frame}[noframenumbering]{Appendix: Government Budget Dynamics}
  Starting from the nominal budget constraint:
  \[
    B_{t+1} = B_t(1+i_{t+1}) + G_{t+1} - T_{t+1}
  \]

  In ratios to GDP ($b_t \equiv B_t/Y_t$):
  \[
    b_{t+1} = b_t\,\frac{1+i_{t+1}}{1+g_{Y,t+1}} + s_{t+1} - \mathcal{T}_{t+1}
  \]
  where $s_t = G_t/Y_t$ and $\mathcal{T}_t = T_t/Y_t$.

  \medskip
  Linearised:
  \[
    \hat{b}_{t+1} = \underbrace{\frac{1+\bar{\imath}}{1+\bar{g}_Y}}_{\rho_b}\,\hat{b}_t + \frac{\bar{\imath}}{1+\bar{g}_Y}\,\hat{\imath}_t - \frac{\bar{g}_Y}{1+\bar{g}_Y}\,\hat{g}_{Y,t+1} + \frac{\bar{s}}{\bar{b}}\,\hat{s}_{t+1} - \frac{\bar{\mathcal{T}}}{\bar{b}}\,\hat{\mathcal{T}}_{t+1}
  \]

  Extension: replace lump-sum rebate of tariff revenue with $G_{ki,t}$ financing, creating a feedback loop between trade and fiscal policy.
\end{frame}

% ------ A9: UNCONSTRAINED FISCAL RULE ------

\begin{frame}[noframenumbering]{Appendix: Optimal Fiscal Rule (Unconstrained Budget, $\sigma>1$, $\kappa>0$)}
  \small
  \[
    g_{k,t} = \frac{H_k}{X_k}\,y_{k,t} + \frac{J_k}{X_k}\,a_{k,t} - \frac{\theta}{\lambda_k X_k}\,\pi_{k,t} + \frac{\sigma{-}1}{(1{-}\chi)^{1/\sigma}X_k}\sum_j\mu_j\lambda_j\,\phi^\pi_{j,t}
  \]
  \begingroup\scriptsize
  \begin{align*}
    H_k &= \chi_k^* + 1 + \varphi + (\sigma{-}1)\tfrac{\omega_{c,k}}{1{-}\chi_k} - \varphi\lambda_k\,\tfrac{1+\chi_k^*+\varphi+\frac{1}{\lambda_k(1-\chi_k)}}{\lambda_k\varphi + \frac{\kappa}{\theta-1}\frac{\lambda_k}{1-\chi_k}} \\[2pt]
    J_k &= -(1{+}\varphi)\left(1 - \lambda_k\,\tfrac{1+\chi_k^*+\varphi+\frac{1}{\lambda_k(1-\chi_k)}}{\lambda_k\varphi + \frac{\kappa}{\theta-1}\frac{\lambda_k}{1-\chi_k}}\right) \\[2pt]
    X_k &= \chi_k^* + (\sigma{-}1)\chi_k^*\omega_{c,k}\chi_k + \bigl(1{+}(\sigma{-}1)\omega_{g,k}\bigr)\Bigl(\lambda_k + \tfrac{1+\chi_k^*+\varphi}{\lambda_k\varphi + \frac{\kappa}{\theta-1}\frac{\lambda_k}{1-\chi_k}}\Bigr) \\[2pt]
    \phi^\pi_{k,t} &= \bigl(-\varphi\,y_{k,t} - g_{k,t}(1{+}(\sigma{-}1)\omega_{g,k}) + (1{+}\varphi)\,a_{k,t}\bigr)\bigl(\lambda_k\varphi + \tfrac{\kappa}{\theta-1}\tfrac{\lambda_k}{1-\chi_k}\bigr)^{-1}
  \end{align*}
  \endgroup

  Comparing welfare under this vs.\ the constrained rule quantifies the cost of fiscal inflexibility.
\end{frame}

% ------ A10: UNCONSTRAINED MONETARY RULE ------

\begin{frame}[noframenumbering]{Appendix: Optimal Monetary Rule (Unconstrained Budget)}
  Optimal monetary policy sets a weighted inflation target:
  \[
    \sum_k \frac{\theta(1{-}\chi_k)}{\lambda_k}\,\pi_{k,t} = -\sum_k \mu_k\,\frac{\varphi\,y_{k,t} + g_{k,t}\bigl(1{+}(\sigma{-}1)\omega_{g,k}\bigr) + (1{+}\varphi)\,a_{k,t}}{\lambda_k\varphi + \frac{\kappa}{\theta-1}\frac{\lambda_k}{1-\chi_k}}
  \]

  \begin{itemize}
    \item Inflation weights increase with private-consumption share, decrease with $\lambda_k$.
    \item Government spending enters the target when $\kappa>0$ (government demand affects marginal costs).
  \end{itemize}
\end{frame}

% ------ A11: FIXED VS ENDOGENOUS DETAIL ------

\begin{frame}[noframenumbering]{Appendix: Fixed vs.\ Endogenous Budgets --- Detail}
  \begin{columns}[T]
    \begin{column}{0.47\textwidth}
      \begin{block}{Fully endogenous spending}
        \begin{itemize}
          \item Planner chooses both $\bar{G}_{k,t}$ and $\{G_{ki,t}\}$.
          \item Level and composition respond to shocks.
          \item Unconstrained fiscal benchmark (first-best within spending instruments).
          \item Fiscal rule is \textit{absolute}: each $g_{k,t}$ set independently.
        \end{itemize}
      \end{block}
    \end{column}
    \begin{column}{0.47\textwidth}
      \begin{block}{Fixed aggregate budget}
        \begin{itemize}
          \item $\bar{G}_{k,t}$ exogenous; only composition adjusts.
          \item Fiscal rule is \textit{relative}: gaps wrt residual sector.
          \item Welfare loss from constraint is concentrated in aggregate stabilisation.
          \item Cross-sectional allocation may remain close to first-best.
        \end{itemize}
      \end{block}
    \end{column}
  \end{columns}

  \bigskip
  \textbf{Implications.}\; If the welfare gap is small, a budget-constrained government can approximate first-best outcomes through compositional reallocation alone.  This is the central hypothesis to be tested in the networked model.
\end{frame}

% ------ A12: OPEN QUESTIONS ------

\begin{frame}[noframenumbering]{Appendix: Open Questions}
  \begin{itemize}
    \setlength{\itemsep}{8pt}
    \item \textbf{Network centrality and fiscal allocation.}\; How does a sector's position in the IO network affect its optimal public-good allocation?  The IO weights $\omega_{klij}$ introduce cross-sector spillovers absent in the simple model.
    \item \textbf{Fiscal--trade policy interaction.}\; With tariff revenue financing government purchases, trade policy changes alter the fiscal envelope.  The welfare implications of this feedback are to be characterised.
    \item \textbf{Dimensionality.}\; The Ramsey problem involves $K \times I$ spending instruments.  Practical approaches may require restricting the class of admissible rules.
    \item \textbf{Political economy.}\; The exogenous-budget assumption abstracts from the determination of $\bar{G}_{k,t}$.  Endogenising this would require a political-economy layer.
  \end{itemize}
\end{frame}


% ============================================================
%
%              AUXILIARY: PREPARED RESPONSES FOR Q&A
%
% ============================================================

% ------ Q1: WHY NOT RAMSEY DIRECTLY? ------

\begin{frame}[noframenumbering]{Q\&A: Why Not Derive the Ramsey Problem Directly?}
  \begin{block}{The question}
    Why modify Phillips curves with fiscal instruments rather than solving a full Ramsey taxation problem?
  \end{block}

  \medskip
  \begin{itemize}
    \setlength{\itemsep}{6pt}
    \item The Phillips-curve approach is a \textbf{reduced-form shortcut}: it isolates the cost-push channel of taxation while preserving the existing IO structure of Aguilar et al.\ (2025).
    \item A full Ramsey problem in a $4 \times 44$ networked model is computationally demanding: it requires solving for $K \times I$ optimal tax instruments simultaneously.
    \item The \textbf{research agenda goal} is precisely to move toward the full Ramsey characterisation.  The current extensions are a tractable first step.
    \item La'O \& Tahbaz-Salehi (2024) show that even in simpler networks, the Ramsey problem has a rich structure.  Our approach builds intuition before scaling up.
  \end{itemize}
\end{frame}

% ------ Q2: RELATION TO RUBBO / LA'O & TAHBAZ-SALEHI ------

\begin{frame}[noframenumbering]{Q\&A: Relation to Rubbo (2023) and La'O \& Tahbaz-Salehi (2024)}
  \begin{block}{The question}
    How does the relative allocation rule relate to optimal policy results in production networks?
  \end{block}

  \medskip
  \begin{columns}[T]
    \begin{column}{0.47\textwidth}
      \textbf{Rubbo (2023):}
      \begin{itemize}
        \setlength{\itemsep}{3pt}
        \item Optimal \textit{monetary} policy in production networks.
        \item Divine coincidence fails; price-stability target depends on network topology.
        \item Our work: fiscal policy adds a second instrument that can target sectoral gaps directly.
      \end{itemize}
    \end{column}
    \begin{column}{0.47\textwidth}
      \textbf{La'O \& Tahbaz-Salehi (2024):}
      \begin{itemize}
        \setlength{\itemsep}{3pt}
        \item Optimal monetary policy in production networks (Econometrica).
        \item Network structure shapes the optimal inflation target and creates trade-offs absent in one-sector models.
        \item Our contribution: government \textit{spending} as a complementary fiscal instrument, with an aggregate budget constraint.
      \end{itemize}
    \end{column}
  \end{columns}

  \bigskip
  \textbf{Key distinction:} our relative allocation rule operates under a \textit{fixed fiscal envelope}, a constraint absent in both papers.
\end{frame}

% ------ Q3: TIGHT VS LOOSE BUDGET CONSTRAINT ------

\begin{frame}[noframenumbering]{Q\&A: When Does the Budget Constraint Bind?}
  \begin{block}{The question}
    What happens when $\bar{G}_t$ is close to optimal vs.\ far from it?
  \end{block}

  \medskip
  \begin{itemize}
    \setlength{\itemsep}{6pt}
    \item When $\bar{G}_t$ is \textbf{close to the first-best level}, compositional reallocation suffices: the relative allocation rule can approximate optimal welfare by reshuffling spending across sectors.
    \item When $\bar{G}_t$ is \textbf{far from optimal} (e.g., a deep austerity constraint), the welfare gap grows because the level channel is shut off.  The planner cannot compensate for an insufficient aggregate envelope by reallocating alone.
    \item The \textbf{welfare gap} between fixed and endogenous budgets is therefore increasing in $|\bar{G}_t - G^*_t|$, where $G^*_t$ is the first-best aggregate level.
    \item Quantifying this gap in the $4 \times 44$ networked model is the central objective of the research agenda.
  \end{itemize}
\end{frame}

% ------ Q4: CALIBRATION GRANULARITY ------

\begin{frame}[noframenumbering]{Q\&A: Why 4 Countries and 44 Sectors?}
  \begin{block}{The question}
    How sensitive are results to this level of granularity?
  \end{block}

  \medskip
  \begin{itemize}
    \setlength{\itemsep}{6pt}
    \item The $4 \times 44$ structure follows Aguilar et al.\ (2025), calibrated to the \textbf{OECD ICIO 2019 tables} (Inter-Country Input-Output).
    \item 4 countries: a practical choice balancing model tractability with open-economy realism (e.g., US, EA, CN, RoW).
    \item 44 sectors: the full ISIC Rev.~4 classification available in ICIO (no aggregation required).
    \item \textbf{Robustness considerations:}
    \begin{itemize}
      \item Coarser aggregations (e.g., 10--15 sectors) can be tested by collapsing IO tables.
      \item The key qualitative predictions (relative reallocation, countercyclicality) should survive aggregation.
      \item Quantitative magnitudes (welfare gaps) are likely sensitive to granularity, as network effects depend on the density of the IO matrix.
    \end{itemize}
  \end{itemize}
\end{frame}

% ------ MPhil: PASS-THROUGH PARAMETER KAPPA ------

\begin{frame}[noframenumbering]{Q\&A: The Pass-Through Parameter $\kappa$}
  \textbf{Motivation.}  Cox et al.\ (2024) assume government demand does not affect firms' pricing decisions ($\kappa=0$).  Empirical evidence (Ben Zeev \& Pappa, 2017) suggests fiscal spending does affect prices.

  \bigskip
  \begin{block}{Definition}
    $\kappa \in [0,1]$ controls how much inelastic government demand passes through to marginal cost.  When $\kappa > 0$:
    \begin{itemize}
      \item The Phillips curve steepens: $\lambda'_k = \lambda_k/(1-\delta)$, where $\delta = \frac{\kappa}{\theta-1}\frac{\bar{G}}{\bar{C}}$.
      \item Government spending becomes a \textbf{supply-side instrument}: $g_{k,t}$ enters the Phillips curve directly.
    \end{itemize}
  \end{block}

  \medskip
  \textbf{Nesting:} $\kappa=0$ recovers Cox et al.\ (2024).  At $\kappa=1$ (full pass-through), for $\theta=6$ and $\chi^*_k=1$: $\lambda'_k = 1.25\,\lambda_k$.
\end{frame}

% ------ MPhil: CRRA AND INSURANCE MOTIVE ------

\begin{frame}[noframenumbering]{Q\&A: CRRA Preferences and the Insurance Motive}
  With $\sigma \neq 1$, the welfare objective acquires an additional term (in \red{red}):
  \small
  \begin{align*}
    -\frac{1}{2}\sum_k\mu_k\Biggl(&
      (1{+}\varphi)\,y_{k,t}^2
      + \frac{\theta(1{-}\chi_k)}{\lambda_k}\,\pi_{k,t}^2
      + \chi_k^*\,(g_{k,t}{-}y_{k,t})^2 \\
    & + \red{(\sigma{-}1)}\Biggl[
        (1{-}\chi_k)\,\omega_{c,k}\!\left(\frac{y_{k,t}}{1{-}\chi_k} - \chi_k^*\,g_{k,t}\right)^{\!2}
        + \omega_{g,k}\,\chi_k\,g_{k,t}^2
      \Biggr]\Biggr)
  \end{align*}
  \normalsize

  \medskip
  Fiscal policy now balances \textbf{three motives}:
  \begin{enumerate}
    \setlength{\itemsep}{3pt}
    \item \textbf{Allocation:} match public-good provision to sectoral needs.
    \item \textbf{Stabilisation:} dampen output and inflation gaps.
    \item \textbf{Insurance:} hedge against aggregate consumption risk ($\sigma{-}1$ term).
  \end{enumerate}
\end{frame}

% ------ MPhil: NEGATIVITY CONDITION ------

\begin{frame}[noframenumbering]{Q\&A: Negativity Condition for Countercyclicality}
  Setting $\kappa=1$, the overall coefficient on $y_{k,t}$ in the fiscal rule is:
  \[
    \underbrace{\frac{1}{(\theta{-}1)(1{-}\chi_k)\varphi+1}}_{\text{always } > 0}
    \;\times\;
    \underbrace{\left[\frac{1}{1{-}\chi_k} + \varphi
      + (\sigma{-}1)\frac{\omega_{c,k}}{1{-}\chi_k}
      - \frac{(\theta{-}1)(1{-}\chi_k)\varphi}{(1{-}\chi_k)\lambda_k}\right]}_{\text{sign determines cyclicality}}
  \]

  \medskip
  Government spending is \textbf{countercyclical} ($g_{k,t}$ falls when $y_{k,t}$ rises) if and only if:
  \[
    \sigma \;>\; \left(\frac{(\theta{-}1)(1{-}\chi_k)\varphi}{\lambda_k} - 1 - \varphi(1{-}\chi_k)\right)\frac{1}{\omega_{c,k}} + 1
  \]

  \medskip
  \begin{itemize}
    \item Under \textbf{flexible prices} ($\lambda_k \to \infty$): always satisfied.
    \item Under \textbf{moderate rigidity} ($\alpha_k < 0.3$): satisfied for standard calibrations.
    \item $\kappa < 1$ implies a \textbf{less countercyclical} stance (the pass-through channel weakens).
  \end{itemize}
\end{frame}

% ------ DERIVATION DEFENSE: WELFARE APPROXIMATION ------

\begin{frame}[noframenumbering]{Q\&A: Welfare Approximation --- Full Derivation Steps}
  \small
  \textbf{Step 1.}  Second-order Taylor expansion of $U$ around the efficient steady state:
  \[
    U - \bar{U} \approx \sum_k \mu_k \left[ \bar{C}\,(1{-}\chi)\,\hat{c}_{k,t} + \bar{G}\,\chi\,\hat{g}_{k,t} - \bar{N}_k^{1+\varphi}\,\hat{n}_{k,t} \right] + \text{second-order terms}
  \]

  \textbf{Step 2.}  Use the production function $Y_{k,t} = A_{k,t}N_{k,t}$ and goods market clearing $Y_{k,t} = C_{k,t} + G_{k,t}$ to eliminate $\hat{n}_{k,t}$ and $\hat{c}_{k,t}$:
  \[
    \hat{n}_{k,t} = y_{k,t} - a_{k,t}, \qquad (1{-}\chi_k)\hat{c}_{k,t} = y_{k,t} - \chi_k\,g_{k,t}
  \]

  \textbf{Step 3.}  Calvo price dispersion contributes $\sum_k \frac{\theta}{\lambda_k}\,\pi_{k,t}^2$ (Woodford, 2003, Ch.~6).

  \medskip
  \textbf{Step 4.}  Collecting terms (setting $\sigma=1$, $\kappa=0$):
  \[
    \mathcal{W} \approx -\frac{1}{2}\sum_k \mu_k \left[ (1{+}\varphi)\,y_{k,t}^2 + \frac{\theta(1{-}\chi_k)}{\lambda_k}\,\pi_{k,t}^2 + \chi_k^*\,(g_{k,t} - y_{k,t})^2 \right]
  \]
  where $\chi_k^* \equiv \chi_k/(1{-}\chi_k)$.  The public-good gap arises from the resource constraint binding at the sectoral level.
\end{frame}

% ------ DERIVATION DEFENSE: RELATIVE RULE FOCs ------

\begin{frame}[noframenumbering]{Q\&A: Relative Allocation Rule --- FOC Details}
  \small
  \textbf{Lagrangian.}  Minimise $\mathcal{W}$ subject to $\sum_k \omega_{g,k}\,g_{k,t} = \bar{g}_t$ (aggregate budget):
  \[
    \mathcal{L} = \mathcal{W} + \eta_t\!\left(\bar{g}_t - \sum_k \omega_{g,k}\,g_{k,t}\right)
  \]

  \textbf{FOC for sector $k$:}
  \[
    \frac{\partial \mathcal{W}}{\partial g_{k,t}} = \mu_k\!\left[\chi_k^*\,(g_{k,t} - y_{k,t}) + \frac{\theta(1{-}\chi_k)}{\lambda_k}\,\frac{\partial \pi_{k,t}}{\partial g_{k,t}} + (1{+}\varphi)\,\frac{\partial y_{k,t}}{\partial g_{k,t}}\right] = \eta_t\,\omega_{g,k}
  \]

  \textbf{Eliminating $\eta_t$:}  Equate the FOC for sector $k$ with sector $i$ (the residual):
  \[
    \frac{1}{\omega_{g,k}}\,\frac{\partial \mathcal{W}}{\partial g_{k,t}} = \frac{1}{\omega_{g,i}}\,\frac{\partial \mathcal{W}}{\partial g_{i,t}}
  \]

  Substituting the Phillips curve $\pi_{k,t} = \lambda_k\bigl[(1{+}\varphi{+}\chi_k^*)\,y_{k,t} - \chi_k^*\,g_{k,t}\bigr] + \beta\,\mathbb{E}_t\pi_{k,t+1}$ (using the resource constraint $\hat{c}_{k,t} = (\hat{y}_{k,t} - \chi_k\,g_{k,t})/(1{-}\chi_k)$) and solving for $g_{k,t}$ in terms of $g_{i,t}$, $y_{k,t}$, $\pi_{k,t}$, $y_{i,t}$, $\pi_{i,t}$ yields the relative allocation rule.

  \medskip
  \textit{See \hyperlink{structural-coefficients}{\beamergotobutton{Structural coefficients}} for the full extended form.}
\end{frame}


\end{document}
